\documentclass[12pt]{report}
\title{uous System}
\author{Spencer Sallay}
\setcounter{section}{0}
\setcounter{secnumdepth}{3}
\usepackage[pdftex,bookmarks,colorlinks]{hyperref}
\usepackage{longtable}
\begin{document}
  \maketitle
  \LaTeX{}
  \tableofcontents
  \newpage
  \chapter{Introduction}
  This is a reference of the \verb|uous| byte-code interpreter.  \verb|uous| is a 
  small hobby project that essentially is a very simplistic VM.  The byte-code is
  a simple stack-based language. This is more just to list how the system works, and
  is not really useful.  The document itself isn't very well formatted yet either,
  so it may be difficult to read.

  \section{Design}
  As mentioned before, the system is made up of several components.  One of these
  components is the byte-code interpreter itself, which acts as a way to create the
  system itself.  

  \subsection{Storage}
  There are two files that the interpreter reads in order to run.
  The first of these two files is the file, \verb|root.uo|.  For further reference,
  \verb|.uo| is the extension for any file relating to \verb|uous|.  \verb|root.uo| is
  is the file that is read automatically when booting up the system.  In this file,
  code should be written that is expected to be run every time the system boots up.
  Conversely, the file \verb|mem.uo| is where to write code that is not expected
  to be read immediately.  There exist functions in the byte code that allow
  reading of a number of bytes at a certain position in \verb|mem.uo| and then parsing
  them.  \verb|mem.uo| is for long-term storage, generally.

  The specifics of how to read code from \verb|mem.uo| and how exactly \verb|root.uo|
  is read is explained later.

  \subsection{Interpreter}
  As mentioned before, when the system starts, the file that is immediately read is
  \verb|root.uo|.  More specifically, the \emph{entire} file is read and stored into
  a linked list to be parsed sequentially.  What this means is that \verb|root.uo| is
  read once and then closed, so the file will already be closed by the time the
  program is actually being parsed.  This is what \verb|mem.uo| is for.

  For example, there are a few files on a system.  For this example, a
  file consists of 8 bytes denoting the size of the data contained within itself, and
  the data itself.  Of course, it is probably not ideal to load all files on a
  system every time the system starts, rather it is probably better to load them
  upon request.  It is possible that a certain file will never be called upon in the
  first place, making it a waste if it were to be loaded in \verb|root.uo|.  The
  interpreter has a function that takes as parameters the location (in bytes) where
  the interpreter should start reading and the amount of tokens (size variable) that
  should be read in \verb|mem.uo|.

  Finally, there is a third file called \verb|dump.uo|.  For now, this file will be
  ignored, as it is just a file to transfer output from the interpreter to
  \verb|mem.uo|.

  The actual syntax of \verb|uous| is very simple.  The following examples are
  written using big-endian notation.  All commands in \verb|uous| are represented
  by a byte.  For example, \verb|0| is the command to push an integer value to the
  stack.  The following is a more comprehensive list of all of the commands:
  \begin{tabular}{ r | c || l }
  0 & [8-bytes] & push integer (64-bit signed) \\
  1 & [8-bytes] & push float (64-bit) \\
  2 & [0-bytes] & create label \\
  4 & [1-byte] & system/function call \\
  \end{tabular}

  There is one final thing to mention about the language itself, and it is the idea
  of labels.  Like in many assemblers (e.g. x86), there exist labels which can be
  used with commands like \verb|jmp| to "jump" to a different spot in a program and
  continue reading.  There is also \verb|call|, which is discussed in the reference.
  When the label command is called, the position of the program is stored in a list
  of other labels\footnote{Note that this storage happens during the reading phase
  of program running, and not during parsing}.

  \chapter{Reference}
  The following table is a reference for all of the functions in the language.
  Functions can be called using the system/function call command (e.g.
  [04][02] is an example, where the brackets separate the two bytes for clearer
  reading).  All of the functions affect the stack.  Note that when \emph{Nothing} is
  used to describe an input/output, this is ignoring side-effects.

  \begin{longtable}{ l | l | l || p{6cm} }
  Name & In & Out & Description \\
  Call & \verb|label| & \emph{Nothing} & This functions exactly as \verb|jmp| does,
  but with the side-effect of adding the position jumped from to a reference stack.
  The accompanying function, \verb|ret|, uses this reference stack. \\
  Return & \emph{Nothing} & \emph{Nothing} & \verb|ret| takes the pointer at the
  top of the reference stack and jumps to it.  Usually, this pointer returns the
  program to a position that was previously jumped from. \\
  Jump & \verb|label| & \emph{Nothing} & This takes a label as represented by an
  8-byte integer and jumps to the pointer associated with that label. \\
  + - * / & 2 (\verb|int| or \verb|float|) & Same type as inputs &
  These are some of the standard mathematical operations.  They act as expected with
  no side-effects. \\
  MEM & \verb|int|:n \verb|int|:m & \emph{Nothing} & This reads an amount of bytes
  (\emph{m}) from a certain position (\emph{n}) of \verb|mem.uo| and parses
  them.  This newly parsed code is appended to the end of the storage of the
  program. \\
  Quit & \emph{Nothing} & \emph{Nothing} & Quits the program. \\
  Dump & \emph{Nothing} & \emph{Nothing} & This is an important command that dumps
  the entirety of the contents of the stack into \verb|dump.uo|.  Once the program
  quits, The contents of \verb|dump.uo| are appended to \verb|mem.uo|.  This is a way
  to save data long-term during a session. \\
  Lift & \verb|int|:x & \emph{Special} & This function takes the item \emph{x}
  places from the top of the stack (\emph{x} itself is not included on the stack) and
  brings it to the front.  This moves the actual position, and does not copy the
  value to the front. \\
  Drop & \verb|any| & \emph{Nothing} & This removes the value at the top of the
  stack. \\
  Not & \verb|int| & \verb|int| & If the parameter given is \verb|0|, then
  \verb|1| is returned.  If given any non-zero value, \verb|0| is returned\footnote{0
  and 1 act as the two boolean values.} \\
  JEZ & \verb|int|:l \verb|int|:c & \emph{Nothing} & This is the conditional jump, and
  only performs a jump to label \emph{l} given that integer \emph{c} is zero. \\
  \end{longtable}
\end{document}
