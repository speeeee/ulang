\documentclass[12pt]{report}
\title{ulang Byte-code \& Assembler}
\author{Spencer Sallay}
\setcounter{section}{0}
\setcounter{secnumdepth}{3}
\usepackage[pdftex,bookmarks,colorlinks]{hyperref}
\begin{document}
  \maketitle
  \tableofcontents
  \newpage
  \chapter{Introduction}
  This is a complete\footnote{While this is the goal, the document is \emph{not} 
  complete yet.} reference of the ulang byte-code interpreter and the
  accompanying assembler.
  \section{Purpose}
  ulang is just a small project that I am working on.  There really isn't a set purpose
  for ulang yet.  The ideal, however, is to create a quick and flexible byte-code
  interpreter.  Prior to this, I had created a small LISP interpreter that shares a few 
  similarites with this.  ulang is mainly a project to counteract some of the issues with
  that LISP interpreter.  Aside from that, the project does not necessarily have a set
  goal yet.
  \chapter{Tutorial}
  Before the actual reference is given, there will first be a small tutorial on how to
  use the assembler at a less specific level.  This tutorial will cover a few essential
  opcodes (e.g. \verb|main|, \verb|terminate|, \verb|push|).  While these opcodes
  will be covered, the specifics of each code is explained in the language reference.
  It should be mentioned that in this document, any text with a monospace font is
  either a program or a command-line interaction.  The document will refer to the
  command-line's imput with a \verb|>| preceeding the command.  An empty line
  in the command-line is the output of an executed command.
  \newpage
  \section{A First Program}
  The following is an example of a program written in u-Assembly that prints the number,
  \verb|1|: 
  \begin{verbatim}
	main
	push 1
	push 0
	out_s
	terminate
  \end{verbatim}
  The above text can be copied in to a file with extension \verb|.usm|.  This is
  probably one of the most basic u-Assembly programs.  Running the program can be done
  by typing the following commands:
  \begin{verbatim}
	> assembler name name
	> binterpreter name
	1
  \end{verbatim}
  The example uses the name \verb|name.usm| as the example file.  Essentially, the
  assembler takes an input file with extension \verb|.usm| and outputs a file with the
  corresponding byte-code with extension \verb|.uo|.  The interpreter then reads the
  byte-code file and executes it.  The file here, for example, prints then number, 
  \verb|1|.
  
  Now to go through the file line-by-line and explain what is happening:
  \begin{verbatim}
	main
  \end{verbatim}
  This line essentially defines the entry point for which the interpreter will begin at.
  The interpreter reads each instruction sequentially, and this is the starting point.
  \begin{verbatim}
	push 1
  \end{verbatim}
  Like most assemblers, u-Assembly has a virtual stack that the programmer can push or
  pop things from.  This instruction pushes the number, \verb|1|, to the top of the
  stack.  Later, the number will be popped for use by the instruction, \verb|out_s|.
  \begin{verbatim}
	push 0
  \end{verbatim}
  This is probably the most strange line.  While the instruction itself is clear, the
  reason for it might not be as clear.  The number will be explained when explaining
  \verb|out_s|.
  \begin{verbatim}
	out_s
  \end{verbatim}
  This is the instruction that prints the number at the top of the stack.  However, there
  is a bit more to this.  As mentioned before, two numbers were pushed to the stack.  The
  first number, \verb|1|, is the actual number to be printed.  The second number, 
  \verb|0|, is actually the type of the number to be printed.  \verb|0| corresponds with
  the type, \verb|int|, meaning that the type of what is to be printed is of type,
  \verb|int|.  From this, the two numbers are popped and the \emph{integer}, \verb|1|,
  is printed.
  \begin{verbatim}
	terminate
  \end{verbatim}
  This line is very simple in that when the interpreter reads this instruction, it quits.
  Note that this instruction is necessary to use or else the program will run infinitely.
  The instruction is mainly there for when the intended end of a program is not at the
  end of the corresponding byte-code file.
  \newline
  \section{Using labels and jumps}
  Another one of the essential instructions is \verb|label|.  This instruction basically
  sets a point in the sequence of instructions that the instruction, \verb|ojmp|, can
  go to.  The following is an example of this:
  \begin{verbatim}
	label print_int
	push 0
	out_s
	terminate

	main
	push 1
	push print_int
	ojmp
  \end{verbatim}
  Something to be noted is that this program is actually functionally the same as the
  last program, printing the number, \verb|1|.  Instead of just calling \verb|out_s|,
  the program instead jumps to a label that does it.  As can be seen, the syntax for
  jumping is done by pushing the label to jump to and calling \verb|ojmp|.  Note
  that the \verb|1| pushed previously is still on the stack when it jumps to the label.
  From there, the program carries on just like it did last time.  The only issue is that
  this isn't really the use for \verb|ojmp|, as just changes the location of the program
  for no reason.  A better example would use the instruction \verb|ocall|\footnote{\emph{ocall} is not actually implemented yet}.  Here is the
  altered program using \verb|ocall|:
  \begin{verbatim}
	label print_int
	push 0
	out_s
	return

	main
	push 1
	push print_int
	ocall
	terminate
  \end{verbatim}
  This program is the same as the last one, but instead of terminating immediately after
  the call, it actually returns to the point from which it was called.  This doesn't
  really do much as it just terminates afterword, but the idea of \verb|ocall| is that
  it simulates the behavior of a function in higher level languages.  There is also two
  conditional jumps, called \verb|ojez| and \verb|ojns|.  The first pops a number from
  the stack and checks if it is equal to zero.  If it is, then the jump happens; if not
  the jump fails and the program continues as normal.  \verb|ojns| is similar, except it
  just checks if the stack is empty (that is, empty besides the label that was pushed
  to jump to).  The specifics of these instructions are explained in the reference.
  \section{Essential instructions}
  \begin{itemize}
  \item Will be done later
  \end{itemize}
  \chapter{Syntax}
  \section{u-Assembly syntax}
  The syntax of u-Assembly is very simple, and only contains a few rules.  First, all
  tokens are separated by space characters.  Unlike other assemblers, all space
  characters are the same (e.g. newline, tab, etc.).  Despite this, it is usually
  better to designate a line for each op-code (and associated arguments with that
  op-code).  Op-codes themselves are comprised of any characters that are not
  spaces and they must not begin with a number.  There are also special times where
  symbols are input for other purposes.  For example, the \verb|import| code requires
  a symbol representing the name of the file to refer to.  When printing out to the
  byte-code file, what is written is the op-code, the size of the symbol, and
  the symbol itself represented as a list of characters.  The last type of token is
  the number.  Normally, numbers are read as integers, but by suffixing them with a
  certain character (\verb|f|,\verb|c|,\verb|l|), they will refer to that type
  (\emph{float},\emph{char}, and \emph{long}, respectively).
  \subsection{Byte-code syntax}
  The syntax of the byte-code interpreter is also very simple.  All op-codes are
  comprised of a single byte.  Optionally, a literal of a certain size can
  accompany this op-code and act as the argument for the op-code.  As mentioned before
  op-codes requiring symbols have the symbol's size and the symbol itself as the
  arguments.
  \chapter{Byte-code and u-Assembly reference}
  The rest of this text will contain information about every byte-code in the 
  byte-code representation as well as its u-Assembly counterpart.  It should be noted
  that u-Assembly is \emph{not} a one-to-one translation of the byte-code.  It is,
  however, very close to one-to-one, and whenever there is a difference, it will be
  noted in the reference.  There are also a few codes that are either unused or
  merged with others.  The format of the actual names will be as follows:
  op-code : name.  The op-codes are written in hexadecimal.
  \section{Miscellaneous}
  \subsection{\emph{00, 01, 02, 03 : push}}
  \textbf{Byte-code}: The four byte-codes above correlate with different types of
  items to be pushed: \verb|00| is an integer, \verb|01| is a float, \verb|02| is a
  char, and \verb|03| is a long integer.  These are the four basic types in the system.
  The stack itself is fairly simple, as it is represented by a linked-list, where
  \verb|push| simply adds whatever is to be pushed to the end of the stack.\newline
  \textbf{Assembler}: As seen above, all four op-codes are put into a single name in
  u-Assembly.  The actual op-code that is called is found by whatever type of
  element it is given.  There is one other case, however.  If a symbol that is
  supposed to represent a label is pushed, then it is written as two integers being
  pushed to the stack.  The two integers represent the file and place that the
  label refers to.  See the \verb|label| section for more information on this.
  \newline
  All push functions take a single associated argument.
  \subsection{\emph{19 : pop}}
  \textbf{Byte-code}: This instruction is the same as the \verb|drop| word in
  higher-level stack languages in that it simply pops from the stack and drops what
  is popped.\newline
  \textbf{Assembler}: \verb|pop| is a one-to-one representation.
  \subsection{\emph{26 : swap}}
  \textbf{Byte-code}: This instruction is the same as the \verb|swap| word in
  higher-level stack languages.  It swaps the positions of the two top items of
  the stack.  The instruction takes two arguments from the stack.
  \textbf{Assembler}: \verb|swap| is a one-to-one representation.
  \subsection{\emph{27 : sref}}
  \textbf{Byte-code}: This instruction pushes to the stack the element that is
  \emph{n} elements down in the stack where \emph{n} is an element on the stack to be
  used.  Note that the number excludes itself when referring to the stack.  Also, the
  element pushed does not delete the old one in the stack.
  \textbf{Assembler}: \verb|sref| is a one-to-one representation.
  \subsection{\emph{29 : arith}}
  \textbf{Byte-code}: This instruction performs different arithmatic operations
  depending on its arguments.  From bottom to top, the op-code takes 4
  arguments: the left number, the right number, the operator, and the type.  The
  operator is a number that corresponds with five different operators: 0 – add,
  1 – subtract, 2 – multiply, 3 – divide, 4 – exponent.  A single item is pushed back
  to the stack which is the result of the operation.
  \textbf{Assembler}: \verb|arith| is a one-to-one representation.

  \section{I/O op-codes}
  \subsection{\emph{10 : in}}
  \textbf{Byte-code}: \verb|in| is essentially the same as the C function, getchar(),
  where a single character is retrieved from \emph{stdin} and pushed to the stack.
  \newline
  \textbf{Assembler}: \verb|in| is a one-to-one representation of its byte-code
  counterpart.
  \subsection{\emph{1A : out\_s}}
  \textbf{Byte-code}: \verb|out| is a code that prints a single numeric value.  It
  takes two arguments from the stack.  The first argument that it takes is the 
  type of the item that is to be printed.  The second argument is the argument 
  to be printed.  The code pushes nothing back to the stack.\newline
  \textbf{Assembler}: The instruction, \verb|out|, is a one-to-one representation of
  its byte-code counterpart.

  \section{Labels and jumps}
  \subsection{\emph{11 : label}}
  \textbf{Byte-code}: \verb|label| is a code that is used to define places in the code
  for which the program will jump to when the time comes.  \verb|label| has an
  associated integer that represents the instruction number that it is referring to.
  This integer is pushed to an array of other existing labels, where each label is
  defined by two integers: the file that the label is referring to and the actual
  place in that file the label is referring to.\newline
  \textbf{Assembler}: The u-Assembly version of \verb|label| is somewhat different.
  The most apparent difference is that instead of just an integer, \verb|label| takes
  a symbol, representing the name of the label.  This symbol is pushed to an array
  similar that found in the byte-code interpreter.  Another side-effect is that,
  as mentioned before in the \verb|push| section, when referring to a label, it
  will instead translate to pushing the two integers that make up that label.
  \newline
  Information on how the two integers actually come into play is in the sections for
  \verb|ojmp|, \verb|ocall|, etc.
  \subsection{\emph{21 : return}}
  \textbf{Byte-code}: This code goes to the point in the program referred to by what
  is at the top of the call stack.  After returning to that point, the top of
  the call stack is dropped.  More information on the call stack is in the sections for
  \verb|ojmp|, \verb|ocall|, etc.
  \textbf{Assembler}: \verb|return| is a one-to-one representation.

  \section{Importing other files and linking libraries}
  \subsection{\emph{28 : link}}
  \textbf{Byte-code}: \verb|link| is the main instruction used to interact with
  dynamically linked libraries and the main way to interact with C functions.  More
  details on the actual linking in the C intefacing section.  \verb|link| takes a
  symbol as its only argument, the symbol being the name of the library to be linked
  (note that the library's extension is needed).  The library is pushed to the stack
  for which functions can be pulled from.
  \textbf{Assembler}: The instruction works like any other one that takes a symbol as
  its argument.
  \subsection{\emph{2D : import}}
  \textbf{Byte-code}: This instruction essentially imports another byte-code file.  It
  starts reading the file, only looking for labels and other imports.  Labels found are
  pushed to the label array, with the integer representing the file being unique to
  that file.  The rest will be explained as the feature is confirmed to be fully
  functional.
  \subsection{\emph{2E : lfun}}
  \textbf{Byte-code}: This instruction is used to extract functions from already
  imported dynamically linked libraries.  It takes two arguments from the stack:
  the library to be extracted from and an integer representing the amount of inputs
  the function takes.  It also has an associated symbol argument which is the name
  of the function to be extracted.  The function is then pushed to an array of
  functions from different libraries.  The library itself remains on the stack.
  \textbf{Assembler}: The instruction works like any other that requires a symbol
  as its argument.
  \subsection{\emph{2F : done}}
  \textbf{Byte-code}: This instruction is specifically significant when being read
  by the program when importing a file.  The instruction tells the interpreter to
  stop reading the file.  It is necessary for any file to be imported.
  \textbf{Assembler}: \verb|done| is a one-to-one representation.

  \section{Essential op-codes}
  \subsection{\emph{18 : terminate}}
  \textbf{Byte-code}: This instruction essentially quits the program when read.  It
  is one of the two instructions that are necessary in a u-Assembly program (the
  other being \verb|main|).  \verb|terminate| takes no arguments and, since it quits
  the program, returns nothing to the stack.\newline
  \textbf{Assembler}: \verb|terminate| is a one-to-one representation.
  \subsection{\emph{1C : main}}
  \textbf{Byte-code}: This instruction essentially represents the point at which the
  program will start executing.  This allows for there to be function-style labels.
  \newline
  \textbf{Assembler}: \verb|main| is a one-to-one representation.
\end{document}
