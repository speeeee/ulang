\documentclass[12pt]{report}
\title{ulang Byte-code \& Assembler}
\usepackage[pdftex,bookmarks,colorlinks]{hyperref}
\begin{document}
  \maketitle
  \tableofcontents
  \newpage
  \section{Introduction}
  This is a complete\footnote{While this is the goal, the document is \emph{not} 
  complete yet.} reference of the ulang byte-code interpreter and the
  accompanying assembler.
  \subsection{Purpose}
  ulang is just a small project that I am working on.  There really isn't a set purpose
  for ulang yet.  The ideal, however, is to create a quick and flexible byte-code
  interpreter.  Prior to this, I had created a small LISP interpreter that shares a few 
  similarites with this.  ulang is mainly a project to counteract some of the issues with
  that LISP interpreter.  Aside from that, the project does not necessarily have a set
  goal yet.
  \section{Tutorial}
  Before the actual reference is given, there will first be a small tutorial on how to
  use the assembler at a less specific level.  This tutorial will cover a few essential
  opcodes (e.g. \verb|main|, \verb|terminate|, \verb|push|).  While these opcodes
  will be covered, the specifics of each code is explained in the language reference.
  It should be mentioned that in this document, any text with a monospace font is
  either a program or a command-line interaction.  The document will refer to the
  command-line's imput with a \verb|>| preceeding the command.  An empty line
  in the command-line is the output of an executed command.
  \newpage
  \subsection{A First Program}
  The following is an example of a program written in u-Assembly that prints the number,
  \verb|1|: 
  \begin{verbatim}
	main
	push 1
	push 0
	out_s
	terminate
  \end{verbatim}
  The above text can be copied in to a file with extension \verb|.usm|.  This is
  probably one of the most basic u-Assembly programs.  Running the program can be done
  by typing the following commands:
  \begin{verbatim}
	> assembler name name
	> binterpreter name
	1
  \end{verbatim}
  The example uses the name \verb|name.usm| as the example file.  Essentially, the
  assembler takes an input file with extension \verb|.usm| and outputs a file with the
  corresponding byte-code with extension \verb|.uo|.  The interpreter then reads the
  byte-code file and executes it.  The file here, for example, prints then number, 
  \verb|1|.
  
  Now to go through the file line-by-line and explain what is happening:
  \begin{verbatim}
	main
  \end{verbatim}
  This line essentially defines the entry point for which the interpreter will begin at.
  The interpreter reads each instruction sequentially, and this is the starting point.
  \begin{verbatim}
	push 1
  \end{verbatim}
  Like most assemblers, u-Assembly has a virtual stack that the programmer can push or
  pop things from.  This instruction pushes the number, \verb|1|, to the top of the
  stack.  Later, the number will be popped for use by the instruction, \verb|out_s|.
  \begin{verbatim}
	push 0
  \end{verbatim}
  This is probably the most strange line.  While the instruction itself is clear, the
  reason for it might not be as clear.  The number will be explained when explaining
  \verb|out_s|.
  \begin{verbatim}
	out_s
  \end{verbatim}
  This is the instruction that prints the number at the top of the stack.  However, there
  is a bit more to this.  As mentioned before, two numbers were pushed to the stack.  The
  first number, \verb|1|, is the actual number to be printed.  The second number, 
  \verb|0|, is actually the type of the number to be printed.  \verb|0| corresponds with
  the type, \verb|int|, meaning that the type of what is to be printed is of type,
  \verb|int|.  From this, the two numbers are popped and the \emph{integer}, \verb|1|,
  is printed.
  \begin{verbatim}
	terminate
  \end{verbatim}
  This line is very simple in that when the interpreter reads this instruction, it quits.
  Note that this instruction is necessary to use or else the program will run infinitely.
  The instruction is mainly there for when the intended end of a program is not at the
  end of the corresponding byte-code file.
  \newline
  \subsection{Using labels and jumps}
  Another one of the essential instructions is \verb|label|.  This instruction basically
  sets a point in the sequence of instructions that the instruction, \verb|ojmp|, can
  go to.  The following is an example of this:
  \begin{verbatim}
	label print_int
	push 0
	out_s
	terminate

	main
	push 1
	push print_int
	ojmp
  \end{verbatim}
  Something to be noted is that this program is actually functionally the same as the
  last program, printing the number, \verb|1|.  Instead of just calling \verb|out_s|,
  the program instead jumps to a label that does it.  As can be seen, the syntax for
  jumping is done by pushing the label to jump to and calling \verb|ojmp|.  Note
  that the \verb|1| pushed previously is still on the stack when it jumps to the label.
  From there, the program carries on just like it did last time.  The only issue is that
  this isn't really the use for \verb|ojmp|, as just changes the location of the program
  for no reason.  A better example would use the instruction \verb|ocall| 
  \footnote{\emph{ocall} is not actually implemented yet}.  Here is the
  altered program using \verb|ocall|:
  \begin{verbatim}
	label print_int
	push 0
	out_s
	return

	main
	push 1
	push print_int
	ocall
	terminate
  \end{verbatim}
  This program is the same as the last one, but instead of terminating immediately after
  the call, it actually returns to the point from which it was called.  This doesn't
  really do much as it just terminates afterword, but the idea of \verb|ocall| is that
  it simulates the behavior of a function in higher level languages.  There is also two
  conditional jumps, called \verb|ojez| and \verb|ojns|.  The first pops a number from
  the stack and checks if it is equal to zero.  If it is, then the jump happens; if not
  the jump fails and the program continues as normal.  \verb|ojns| is similar, except it
  just checks if the stack is empty (that is, empty besides the label that was pushed
  to jump to).  The specifics of these instructions are explained in the reference.
  \subsection{Essential instructions}
  
\end{document}
